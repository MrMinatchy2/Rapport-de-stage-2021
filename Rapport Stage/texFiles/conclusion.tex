\hypertarget{conclusion}{%
\chapter{Conclusion}\label{conclusion}}

\section{Rappel de la problèmatique}

Notre problèmatique à la base était la résolution de sudoku Standart ce que nous avons reussi. Nous avons donc pour avoir une autre vision du problème nous avons choisi de généraliser celui-ci. Ce qui nous a ouvert à des pistes d'améliorations dont nous n'aurions pas eu besoin dans le cas de la résolution de sudoku standart.

\section{Réponse apportées}
Nous avons en ce qui concerne la résolution de sudoku standart deux programmes permettant leurs résolution quand aà la généralisation de ceux-ci nous avons un programes parfaitement fonctionnel permettant sa résolution via Cplex.
\section{Piste d'amélioration}
Nous pourrions pour voir les limites de l'algorithme du Cplex l'utiliser sur un grand volume de sudoku et voir le temps mis pour tous les résoudre en une seule fois. Nous pourrions essayer avec des sudokus d'une taille extrème pour en vérifier la consistence.\newline Comme dit précédemment en ce qui concerne la déduction des valeurs nous pouvons éviter beaucoup de test inutiles grace à cela pour amélioré notre algorithme utilisant le backtracking.\newline
Nous pouvons aussi pour les tests sur les sudokus de taille standart tester nos algorithme sur les classes de sudoku décrite dans cette article:\newline
\cite{Differents}
\newline
Nous expliquant que tout les sudoku possibles peuvent être obtenu par plusieurs actions élémentaires à partir de 5472730538 grilles de sudoku que nous pouvons qualifier d'élémentaires.
Cela nous permettrait d'affirmer de manière solide que nos algorithme fonctionnent dans tout les cas possibles.

\section{Les apports du stage}

\subsection{les apports à l'entreprise}
Grace à ce Stage les chercheurs auront une meilleur idée de l'utilisation de Cplex sur python pour la résolution de leurs problèmes et grace à notre algorithme utilisant le backtracking nous pouvons appuyer que l'utilisation des algorithmes de Cplex est bien plus efficaces que les méthodes que nous pouvons coder de façon artisanal sans y consacrer une très grande quantité de connaissances.

\subsection{les apports personels}

Avant ce stage je n'avais aucune connaissance en résolution de problème d'optimisation linéaire. Grace à ce stage je connais maintenant un de ses outils de résolution les plus puissant. J'ai aussi pu apprendre plusieurs termes de mathématique combinatoire que je n'aurais jamais connu autrement. J'ai maintenant la connaissance de plusieurs algorithme de résolution de problème de satisfaction de contrainte. J'ai nmaintenant aussi les connaissances basique nécéssaire quant à la programmation d'interface graphique avec Qt.
