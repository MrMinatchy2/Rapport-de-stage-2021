\hypertarget{Introduction}{%
\chapter{Introduction}\label{Introduction}}

\section{Présentation de la structure
d'accueil}

Durant la période de mon stage , j'ai été accueilli au
\textbf{Laboratoire de Mathématiques Informatique et Application
(LAMIA)} de l'Université des Antilles (UA).

Pour présenter cette structure, il me faut tout d'abord présenter
l'université à laquelle il est rattaché.

\hypertarget{luniversite-des-antilles}{%
\subsection{L'université des Antilles}\label{luniversite-des-antilles}}

Bien que ce soit l'université dans laquelle j'ai fait toutes mes études,
voici quelques chiffres que je ne connaissais pas et qui donnent la
mesure de sa taille :

L'Université des Antilles s'organise autour deux pôles universitaires
régionaux autonomes : le « Pôle Guadeloupe » et le « Pôle Martinique ».

Sur ces pôles, l'Université assure des missions d'\emph{enseignement} et
de \emph{recherche}, assistées par des \emph{administratifs et des
techniciens}.

\hypertarget{administration-et-personnel-technique}{%
\subsubsection{Administration et personnel
technique}
\label{administration-et-personnel-technique}}

l'UA emploie 414 Administratifs et Techniciens (environ 200 personnes
pour l'administation centrale et 100 répartis sur chaque pôle)

\hypertarget{enseignements}{%
\subsubsection{Enseignements}\label{enseignements}}

L'UA délivre des diplomes de la licence au doctorat dans de nombreux
domaines. Au total, cela représente :

\begin{itemize}
\tightlist
\item
  484 enseignants-chercheurs (environ 240 pour chaque pôle)
\item
  12 000 étudiants (environ 7000 pour la Guadeloupe , 5000 pour la
  Martinique)
\end{itemize}

Pour l'informatique, cela représente : - autour de 20
enseignants-chercheurs - autour de 120 étudiants

\hypertarget{le-lamia}{%
\subsection{Le LAMIA}\label{le-lamia}}

Le \textbf{Laboratoire de Mathématiques Informatique et Application
(LAMIA)}, comme son nom l'indique, se concentre sur les recherches en
informatiques et mathématiques.

Il compte une soixantaine de membres (Professeurs des Universités,
Maitres de Conférences, ATER, Doctorants) répartis sur deux pôles
(Guadeloupe et Martinique) au sein de trois équipes internes :

\begin{itemize}
\tightlist
\item
  Equipe
  \href{http://lamia.univ-ag.fr/index.php?page=equipe-mathematiques}{\textbf{Mathématiques}
  (analyse variationnelle, analyse numérique, EDP, analyse statistique,
  mathématiques discrètes)} ;
\item
  Equipe Informatique
  \href{http://lamia.univ-ag.fr/index.php?page=equipe-danais}{\textbf{DANAIS}
  : Data analytics and big data gathering with sensors} ;
\item
  Equipe Informatique
  \href{http://lamia.univ-ag.fr/index.php?page=equipe-aid}{\textbf{AID}
  : Apprentissages Interactions Donnees} ;
\end{itemize}

De plus, le LAMIA accueille en son sein un groupe de chercheurs associés
travaillant en Epidémiologie clinique et médecine.



\hypertarget{contexte-general}{%
\section{Contexte général}\label{contexte-guxe9nuxe9ral}}
\subsection{Qu'est ce que le sudoku}

Le sudoku est un jeu représenté par une grille de 81 cases découpées en 9 lignes et 9 colonnes et 9 sous grilles 3 par 3.
Le but du jeu est de remplir chaque ligne avec 9 chiffres allant de 1 à 9 en faisant en sorte qu'il n'y ai pas le même chiffre plusieurs fois sur la même ligne colonne ou dans la même sous grille.

\begin{figure}[!h]
\centering
\includegraphics[width=5cm]{./images/Sudoku_Exemple.png}\label{Sudoku}
\caption{Exemple de Sudoku complet.}
\end{figure}

\hypertarget{Contexte du problème}{%
\section{Contexte du problème}\label{Contexte_du_probleme}}

La résolution de sudoku est un sujet où plusieurs solutions existent et où c'est dans la complexité\footnote{le nombre d'action réalisé durant la résolution} et le temps de calcul des différentes solution que réside la difficulté. Nous pouvons aussi rendre compte de résolution de sudoku avec des règles spécifiques.
\newline
tel que:

\begin{figure}[!h]
\centering
\includegraphics[width=5cm]{./images/Sudoku_Special.png}\label{Sudoku_Special}
\caption{Sudoku où l'on applique la règle de l'unicité des chiffres sur les diagonales.}
\end{figure}

Dans ce mémoire nous allons éssentiellement parler de deux d'entre elles celle de la résolution de sudoku vu sous l'angle d'un problème d'optimisation linéaire grace à l'algorithme du simplex et une autre plus simple celle de l'algorithme du backtraking.


\hypertarget{Méthodologie}{
\section{Méthodologie}\label{Méthodologie}}
\subsection{Outils utilisés}



\subsubsection{Présentation de Python}

\begin{figure}[h]
  \begin{center}
  \includegraphics[width=4cm]{./images/Python_Logo.png}\label{Python}
  \caption{Logo de Python.}
  \end{center}
\end{figure}

Python est un langage de programmation interprété\footnote{\label{interprete}Langage nécéssitant un programme informatique qui joue le rôle d’interface entre le projet et le processeur appellé interpréteur, pour exécuter du code.} qui sera utilisé pour l'ensemble du projet. Nous avons choisi ce langage car il est très propice aux méthodes utilisées et contient toutes les librairies nécéssaires.

\subsubsection{Présentation Qt}

\begin{figure}[h]
  \begin{center}
  \includegraphics[width=2cm]{./images/Qt_logo_2016.png}\label{Qt}
  \caption{Logo de Qt.}
  \end{center}
\end{figure}

QT est une librairie\footnote{\label{librairie}Une librairie est un fichier contenant du code (généralement un ensemble de fonction et classes permettant de faciliter et/ou de réaliser certains programmes)} qui permet la création d'interface graphique en Python.Que nous emploierons pour créer l'interface graphique que l'on utilisera au cours du projet.

\subsubsection{Présentation Cplex}

\begin{figure}[h]
  \begin{center}
  \includegraphics[width=4cm]{./images/cplex.png}\label{Cplex}
  \caption{Logo de Cplex.}
  \end{center}
\end{figure}

Cplex est une librairie\footref{librairie} qui permet la modélisation et la résolution de problème d'optimisation linéaire\footnote{Terme que j'expliciterai plus tard dans mon rapport}.
\newline
\newline
\subsubsection{Présentation de GitHub}
\begin{figure}[h]
  \begin{center}
  \includegraphics[width=3cm]{./images/github.jpg}\label{GitHub}
  \caption{Logo de GitHub.}
  \end{center}
\end{figure}


Nous pouvons définir GitHub comme une plateforme de développement de projet informatique en groupe. Elle simplifie grandement le développement de projets. Elle permet de versioner ses programmes et d'y apporter des modifications en temps réel à plusieurs. Ce rapport ainsi que le code généré durant le projet en plus de ce trouver en annexe, se trouverons aussi sur github à ces addresses:\newline
Rapport de Stage: \url{https://github.com/MrMinatchy2/Rapport-de-stage-2021} \newline
Code généré: \url{https://github.com/MrMinatchy2/Projet-de-Stage} \newline

\subsubsection{Présentation de LaTex}

\begin{figure}[h]
  \begin{center}
\includegraphics[width=2cm]{./images/Latex.png}\label{LaTex}
\caption{Logo de LaTex}
\end{center}
\end{figure}

Nous pouvons dire que LaTex est un langage de traitement de texte tel que le markdown qui permet de mettre en forme notre texte de manière \"scientifique\" cela veut dire que. LaTex permet une faciliter d'écriture des équations et de toutes les écriture mathématiques.Permet de par ses nombreux package une quasi-infinité de possibilitées.
L'utilisation de cet outil permettra une synergie entre ceux-ci car LaTex peut-être utiliser avec un simple bloc-note c'est donc du texte ce qui permet une intéraction facilitée avec GitHub d'ailleurs ce rapport est écrit avec Latex et retrouvable sur GitHub.\newline


\hypertarget{Annonce du plan}{%
\section{Annonce du plan}\label{annonce du plan}}

\subsection{Présentation des stratégies de résolution}

Dans cette première partie je commencerais par vous présenter la première solution de résolution choisie qui est la résolution du sudoku en tant que problème d'optimisation linéaire.\newline
En deuxième grande sous partie de cette section je vous présenterai ce qu'est le backtraking.\newline
En dernière sous partie de cette présentation je vous expliquerai pourquoi le choix de ces deux solutions.
\subsection{Implémentation des stratégies}
La première sous partie de cette grande sous partie commencera avec la modélisation et l'implémentation d'un sudoku en python et l'implémentation de l'interface graphique.\newline
La seconde sera l'implémentation de la méthode résolution utilisant cplex.\newline
Pour finir nous ferons l'implémentation de la méthode utilisant l'algorithme du backtracking.\newline
\subsection{Test du résolveur}
Nous commencerons par établir nos méthodes de test et expliquer la raison du choix de ces test.\newline
Nous continuerons avec l'implémentation de ceux-ci.\newline
Nous finirons par présenter et analyser nos résultats.\newline
\subsection{Conclusion}
Nous terminerons ce rapport par une conclusion où nous rappelerons brièvement la problématique.\newline
Nous ferons le bilan des produits du projet de stage.\newline
Et ferons le bilan des apports du stage.\newline
