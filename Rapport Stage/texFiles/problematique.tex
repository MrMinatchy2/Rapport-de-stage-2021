
\hypertarget{contexte-general}{%
\section{Contexte général}\label{contexte-guxe9nuxe9ral}}
\subsection{Qu'est ce que le sudoku}

Le sudoku est un jeu représenté par une grille de 81 cases découpées en 9 lignes et 9 colonnes et 9 sous grilles 3 par 3.
Le but du jeu est de remplir chaque ligne avec 9 chiffres allant de 1 à 9 en faisant en sorte qu'il n'y ai pas le même chiffre plusieurs fois sur la même ligne colonne ou dans la même sous grille.

\begin{figure}[!h]
\centering
\includegraphics[width=5cm]{./images/Sudoku_Exemple.png}\label{Sudoku}
\caption{Exemple de Sudoku complet.}
\end{figure}

\hypertarget{Contexte du problème}{%
\section{Contexte du problème}\label{Contexte_du_probleme}}

La résolution de sudoku est un sujet où plusieurs solutions existent et où c'est dans la complexité\footnote{le nombre d'action réalisé durant la résolution} et le temps de calcul des différentes solution que réside la difficulté. Nous pouvons aussi rendre compte de résolution de sudoku avec des règles spécifiques.
\newline
tel que:

\begin{figure}[!h]
\centering
\includegraphics[width=5cm]{./images/Sudoku_Special.png}\label{Sudoku_Special}
\caption{Sudoku où l'on applique la règle de l'unicité des chiffres sur les diagonales.}
\end{figure}

Dans ce mémoire nous allons éssentiellement parler de deux d'entre elles celle de la résolution de sudoku vu sous l'angle d'un problème d'optimisation linéaire grace à l'algorithme du simplex et une autre plus simple celle de l'algorithme du backtraking.
