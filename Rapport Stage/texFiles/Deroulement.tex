\hypertarget{Déroulement}{%
\chapter{Déroulement}\label{Déroulement}}

Ce chapitre, le plus volumineux du rapport, décrira l'ensemble des tâches que j'ai eu à effectuer au cours de ces deux mois.


\section{Présentation des stratégies de résolution}
\subsection{Présentation de la résolution avec Cplex}
\subsubsection{Qu'est-ce qu'un problème d'optimisation linéaire}
Nous allons dans cette première partie parler de la résolution de sudoku comme étant un problème d'optimisation linéaire. Mais tout d'abord nous devons définir ce qu'est un problème d'optimisation linéaire.\newline\newline
Par définition un problème d'optimisation linéaire est un problème dont la valeur à obtimiser ainsi que les contraintes qui y seront appliquer peuvent-êtres modélisées sous la forme d'une fonction linéaire cette description n'est pas des plus précises que l'on puisse c'est pour cela que nous allons l'étoffer par un problème connu.\newline\newline
Le problème du brasseur de bière peut être énoncer comme suit:\newline\newline
Un brasseur fabrique 2 types de bières : blonde et brune.\newline
3 ingrédients : maïs , houblon , malt.\newline
Quantités requises par unité de volume:\newline
Bière blonde: 2,5 kg de maïs, 125 g de houblon, 17,5 kg de malt\newline
Bière brune : 7,5 kg de maïs, 125 g de houblon, 10 kg de malt\newline
Le brasseur dispose de 240 kg maïs, 5 kg houblon , 595 kg malt\newline
Prix vente par u.v. : blonde 15 euros , brune 20 euros
Le brasseur veut maximiser son revenu .\newline
Quelle quantité de bières blondes et/ou brunes doit-il produire pour cela ?\newline
\newline
Nous pouvons modéliser le revenu du brasseur comme étant une fonction linéaire tel que:\newline
Soit $x^{1}$ le nombre de volumes d'unité de bière blonde et $x^{2}$ le nombre d'unité de volume de bière brune\newline
Nous avons:\newline\newline
Le revenu que nous de vons maximiser: $R = x^{1}*15 + x^{2}*12$\newline\newline
Les contraintes peuvent êtres représentées comme suit:\newline\newline
La quantité maximale de maïs: $M \geq x^{1}*2,5+x^{2}*7,5$\newline
La quantité maximale de houblon: $H \geq x^{1}*0,125+x^{2}*0,125$\newline
La quantité maximale de malt: $Ma \geq x^{1}*17,5+x^{2}*10$\newline\newline

\subsubsection{Comment résoudre un problème d'optimisation linéaire}

 Maintenant que nous avons vu ce qu'est un problème d'optimisation linéaire et illustré ceci par un exemple nous allons voir comment peut-on le résoudre,plus particulièrement comment avec l'algorithme du simplex en théorie puis avec Cplex.
\subsection{Stratégie de résolution}
\subsection{Implémentation}
